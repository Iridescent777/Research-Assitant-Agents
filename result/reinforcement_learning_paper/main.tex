\documentclass[12pt]{article}
\usepackage[utf8]{inputenc}
\usepackage{amsmath}
\usepackage{amsfonts}
\usepackage{amssymb}
\usepackage{graphicx}
\usepackage{subfig}
\usepackage{textcomp}
\usepackage{hyperref}
\usepackage{cite}
\usepackage{multirow}
\usepackage{geometry}
\usepackage{setspace}
\usepackage{abstract}
\usepackage{titlesec}

% Page setup
\geometry{margin=1in}
\setlength{\parindent}{0pt}
\setlength{\parskip}{6pt}

% Title formatting
\titleformat{\section}{\Large\bfseries}{\thesection}{1em}{}
\titleformat{\subsection}{\large\bfseries}{\thesubsection}{1em}{}

% Hyperref setup
\hypersetup{
    colorlinks=true,
    linkcolor=blue,
    filecolor=magenta,      
    urlcolor=cyan,
    citecolor=blue,
    pdftitle={强化学习:理论与实践},
    pdfauthor={AI研究团队},
    pdfsubject={Research Paper},
    pdfkeywords={research, academic, paper}
}

\begin{document}

% Title page
\begin{titlepage}
    \centering
    \vspace*{2cm}
    {\Huge\bfseries 强化学习:理论与实践\par}
    \vspace{1.5cm}
    {\large AI研究团队\par}
    \vspace{0.5cm}
    {\large \today\par}
    \vfill
    {\large \textit{Automatically generated research paper}\par}
\end{titlepage}

% Abstract
\begin{abstract}
This paper presents research findings and analysis in the field of 强化学习:理论与实践. 
The content has been automatically generated and compiled into a professional academic format.
\end{abstract}

% Table of contents
\tableofcontents
\newpage

% Main content
\section{引言}
强化学习是机器学习的一个重要分支,专注于训练智能体通过与环境交互来学习最优策略。

\section{核心概念}
\subsection{智能体与环境}
在强化学习中,智能体通过采取行动并从环境获得奖励反馈来学习。

\section{主要算法}
\subsection{Q学习}
Q学习是一种无模型的强化学习算法,用于学习行动的价值函数。

\section{应用领域}
强化学习已成功应用于游戏、机器人、自动驾驶等多个领域。

\section{结论}
强化学习为自主决策系统提供了强大的理论基础。


% Bibliography
\bibliographystyle{unsrt}
\bibliography{references}

\end{document}
