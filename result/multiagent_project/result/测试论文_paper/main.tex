\documentclass[12pt]{article}
\usepackage[utf8]{inputenc}
\usepackage{amsmath}
\usepackage{amsfonts}
\usepackage{amssymb}
\usepackage{graphicx}
\usepackage{subfig}
\usepackage{textcomp}
\usepackage{hyperref}
\usepackage{cite}
\usepackage{multirow}
\usepackage{geometry}
\usepackage{setspace}
\usepackage{abstract}
\usepackage{titlesec}

% Page setup
\geometry{margin=1in}
\setlength{\parindent}{0pt}
\setlength{\parskip}{6pt}

% Title formatting
\titleformat{\section}{\Large\bfseries}{\thesection}{1em}{}
\titleformat{\subsection}{\large\bfseries}{\thesubsection}{1em}{}

% Hyperref setup
\hypersetup{
    colorlinks=true,
    linkcolor=blue,
    filecolor=magenta,      
    urlcolor=cyan,
    citecolor=blue,
    pdftitle={测试论文},
    pdfauthor={测试作者},
    pdfsubject={Research Paper},
    pdfkeywords={research, academic, paper}
}

\begin{document}

% Title page
\begin{titlepage}
    \centering
    \vspace*{2cm}
    {\Huge\bfseries 测试论文\par}
    \vspace{1.5cm}
    {\large 测试作者\par}
    \vspace{0.5cm}
    {\large \today\par}
    \vfill
    {\large \textit{Automatically generated research paper}\par}
\end{titlepage}

% Abstract
\begin{abstract}
This paper presents research findings and analysis in the field of 测试论文. 
The content has been automatically generated and compiled into a professional academic format.
\end{abstract}

% Table of contents
\tableofcontents
\newpage

% Main content
\section{测试}
这是一个测试章节。

% Bibliography
\bibliographystyle{unsrt}
\bibliography{references}

\end{document}
